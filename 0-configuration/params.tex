\titulo{Digite o título do trabalho, somente a primeira palavra do título e os nomes próprios com letras iniciais maiúsculas}
\subtitulo{Digite o subtítulo do trabalho, somente os nomes próprios com letras iniciais maiúsculas} 

\autor{Ettore Pureza Leonel Bigi de Aquino}
\autorR{Pureza Leonel Bigi de Aquino, Ettore}

\local{Juiz de Fora}
\dia{dd}
\mes{mm}
\ano{2023}

\orientador[Orientador:]{Bruno Henriques Dias}
\orientadorR{Henriques Dias, Bruno} 
\orientadorTitulo{Prof.}
\orientadorDegree{D.Sc.}

%\coorientador[Coorientador:]{Nome do coorientador}
%\coorientadorR{Sobrenome, Nome do coorientador}
%\coorientadorTitulo{Prof.} 
%\coorientadorDegree{Dr.}

\instituicao{Universidade Federal de Juiz de Fora}
\faculdade{Faculdade de Engenharia}

\area{Sistemas de Potência}

\tipodocumento{2} %% 1-> TCC; 2-> Mestrado; 3-> Doutorado

\ifthenelse{\equal{\inseretipodocumento}{1}}
	{\programa{Coordenação do Curso de Engenharia Elétrica}}
	{\programa{Programa de Pós-Graduação em Engenharia Elétrica}}

\ifthenelse{\equal{\inseretipodocumento}{1}}
	{\objeto{Trabalho de Conclusão de Curso (Graduação)}}
	{\ifthenelse{\equal{\inseretipodocumento}{2}}
	{\objeto{Dissertação (Mestrado)}}
	{\objeto{Tese (Doutorado)}}
	}

\ifthenelse{\equal{\inseretipodocumento}{1}}
	{\natureza{Trabalho de Conclusão de Curso apresentado à \insereprograma ~da \insereinstituicao ~como requisito parcial à obtenção do grau de Bacharel em Engenharia Elétrica. Modalidade: \inserearea.}}
	{\ifthenelse{\equal{\inseretipodocumento}{2}}
	{\natureza{Dissertação apresentada ao \insereprograma ~da \insereinstituicao ~como requisito parcial à obtenção do título de Mestre em Engenharia Elétrica. Área: \inserearea.}}
	{\natureza{Tese apresentada ao \insereprograma ~da \insereinstituicao ~como requisito parcial à obtenção do título de Doutor em Engenharia Elétrica. Área: \inserearea.}}
	}

%% Abaixo, prencher com os dados da parte final da ficha catalografica

\keywordI{Palavra-chave1} %% Digitar, dentro das { }, a primeira palavra-chave.
\keywordII{Palavra-chave2} %% Digitar, dentro das { }, a segunda palavra-chave.
\keywordIII{Palavra-chave3} %% Digitar, dentro das { }, a terceira palavra-chave.

\finalcatalog{1. \inserekeywordI. 2. \inserekeywordII. 3. \inserekeywordIII. I. \insereorientadorR , orient. II. Título.}
\ifnotempty{\inserecoorientador}{
	\finalcatalog{1. \inserekeywordI. 2. \inserekeywordII. 3. \inserekeywordIII. I. \insereorientadorR, orient. II. \inserecoorientadorR, coorient. III. Título.}}